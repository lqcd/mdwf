\documentclass[oneside,openright]{report}
\usepackage[ruled,vlined]{algorithm2e}
\usepackage{color}
\definecolor{darkblue}{cmyk}{1,1,0,0.7}
\usepackage[dvipdfm,colorlinks=true,linkcolor=darkblue]{hyperref}
\newcommand{\note}[1]{$[\![$NB: #1$]\!]$}
\newcommand{\DontPrintSemicolon}{\dontprintsemicolon}
\setlength{\parindent}{0pt}
%\setlength{\topmargin}{-40pt}
%\setlength{\oddsidemargin}{-16pt}
%\setlength{\evensidemargin}{-16pt}
%\setlength{\textwidth}{522pt}
%\setlength{\textheight}{700pt}
\setcounter{secnumdepth}{3}
\setcounter{tocdepth}{3}

\title{Speeding up the Conjugate Gradient: EigenCG}
\author{Andrew Pochinsky, Sergey Syritsyn}
\catcode`\$=11
\date{$Id: eigen-cg.tex 1307 2010-01-17 21:48:14Z avp $\\
Release working}
\catcode`\$=3

\begin{document}
\maketitle
\thispagestyle{empty}\hbox{}
\vfill
\copyright 2010 Massachusetts Institute of Technology

Permission is hereby granted, free of charge, to any person obtaining
a copy of this software and associated documentation files (the
"Software"), to deal in the Software without restriction, including
without limitation the rights to use, copy, modify, merge, publish,
distribute, sublicense, and/or sell copies of the Software, and to
permit persons to whom the Software is furnished to do so, subject to
the following conditions:

The above copyright notice and this permission notice shall be
included in all copies or substantial portions of the Software.

THE SOFTWARE IS PROVIDED "AS IS", WITHOUT WARRANTY OF ANY KIND,
EXPRESS OR IMPLIED, INCLUDING BUT NOT LIMITED TO THE WARRANTIES OF
MERCHANTABILITY, FITNESS FOR A PARTICULAR PURPOSE AND
NONINFRINGEMENT. IN NO EVENT SHALL THE AUTHORS OR COPYRIGHT HOLDERS BE
LIABLE FOR ANY CLAIM, DAMAGES OR OTHER LIABILITY, WHETHER IN AN ACTION
OF CONTRACT, TORT OR OTHERWISE, ARISING FROM, OUT OF OR IN CONNECTION
WITH THE SOFTWARE OR THE USE OR OTHER DEALINGS IN THE SOFTWARE.
\pagebreak

\tableofcontents
%\vfill
%\pagebreak

\listofalgorithms
%\vfill
%\pagebreak

\chapter{ALGORITHMS}
\section{Conjugate gradient}
The equation
\[
   M^\dagger M \psi = \chi
\]
can be solved by the conjugate gradient method if the condition number of
 $M^\dagger M$ is small enough. Function \texttt{plainCG()} implements the standard CG solver.
\begin{function}
\KwIn{$M$, the matrix}
\KwIn{$\chi$, the right hand side of the linear equation}
\KwIn{$n$, the maximum number of iterations}
\KwIn{$\epsilon$, required precision}
\KwOut{$\psi$, approximate solution}
\KwOut{$r$, final residue}
\KwOut{$k$, number of iterations used}
\KwOut{$z$, was the last $\pi$ a zero vector of $M$?}
\SetKw{Break}{break}
\SetKw{Or}{or}
\SetKw{False}{false}
\SetKw{True}{true}
\DontPrintSemicolon
\Begin{
  $\psi\leftarrow 0$\;
  $\rho\leftarrow \chi$\;
  $\pi\leftarrow \rho$\;
  $r\leftarrow\langle \rho,\rho\rangle$\;
  $\epsilon\leftarrow r \times\epsilon$\;
  $k\leftarrow 0$\;
  \While{$r > \epsilon$ \and $k < n$}{
    $\omega\leftarrow M\pi$\;
    $\zeta\leftarrow M^{\dagger}\omega$\;
    $c\leftarrow \langle \omega,\omega\rangle$\;
    \lIf{$c = 0$}{
      \Return{$\psi$, $r$, $k$, \True}
    }\;
    $a\leftarrow r/c$\;
    $\rho\leftarrow \rho - a \zeta$\;
    $g\leftarrow \langle \rho,\rho\rangle$\;
    $\psi\leftarrow \psi + a \pi$\;
    $k\leftarrow k + 1$\;
    \lIf{$ g<\epsilon$}{ \Return{$\psi$, $g$, $k$, \False}}\;
    $b\leftarrow g/r$\;
    $r\leftarrow g$\;
    $\pi\leftarrow \rho + b \pi$\;
  }
  \Return {$\psi$, $r$, $k$, \False}\;
}

\caption{plainCG($M$, $\chi$, $n$, $\epsilon$)}
\end{function}

\section{Mixed Conjugate Gradient}
On many platforms reducing precision of the operator results in considerable speedup. If one needs the double precision inverter, one can use \texttt{mixedCG()} if implementations of $M$ are available in both precisions. It is convenient to have two separate tolerance levels: one for the full inverter, and another for the inner fast one.

\begin{function}
\KwIn{$M$, the matrix in double precision}
\KwIn{$M_1$, the matrix in single precision}
\KwIn{$\chi$, the right hand side of the linear equation}
\KwIn{$n$, the maximum number of iterations}
\KwIn{$\epsilon$, required total precision}
\KwIn{$\epsilon_1$, required inner precision}
\KwOut{$\psi$, approximate solution}
\KwOut{$r$, final residue}
\KwOut{$k$, number of iterations used}
\KwOut{$z$, was the last $\pi$ a zero vector of $M$?}
\SetKw{Break}{break}
\SetKw{And}{and}
\SetKw{False}{false}
\SetKw{True}{true}
\SetKwFunction{plainCG}{plainCG}
\DontPrintSemicolon
\Begin{
  $\psi\leftarrow 0$\;
  $r\leftarrow \langle\chi,\chi\rangle$\;
  $\epsilon\leftarrow r\times\epsilon$\;
  $k\leftarrow 0$\;
  \While{$k < n$}{
    $\eta\leftarrow\chi - M^\dagger \times M \times \psi$\;
    $r\leftarrow\langle\eta,\eta\rangle$\;
    \lIf{$r<\epsilon$}{\Return{$\psi$, $r$, $k$, \False}}\;
    $\eta\leftarrow\eta/$sqrt$(r)$\;
    $\delta, q, m, z\leftarrow$\plainCG{$M_1$, $\eta$, $n-k$, $\max(\epsilon/r, \epsilon_1)$}\;
    \lIf{$z$ \And $m=0$}{\Return{$\psi$, $r\times q$, $k$, \True}}\;
    $\psi\leftarrow\psi+$sqrt$(r)\times\delta$\;
    $k\leftarrow k + m$\;
  }
  $\eta\leftarrow\chi - M^\dagger \times M \times \psi$\;
  $r\leftarrow\langle\eta,\eta\rangle$\;
  \Return{$\psi$, $r$, $k$, \False}\;
}
\caption{mixedCG($M$, $M_1$, $\chi$, $n$, $\epsilon$, $\epsilon_1$)}
\end{function}
%\end{document}
%%%%%%%%%%%%%%%%%%%%%%%% TODO
\section{EigenCG}
If the condition number of $M^\dagger M$ is large, convergence of the conjugate gradient is slow. In this case one could try to improve the situation by finding small eigenvectors of $M^\dagger M$ and projecting them out of the solver. One variant of this method is given by \texttt{eigenCG}. The low eigen space is managed by $S$ which is manipulated by the set of routines described in the following sections.

Since single precision is 50\% faster than double precision for clover action on the BlueGene,
we use two representations of $M$ and construct the eigen space in single precision. 
It is not strictly necessary, though.

\begin{function}
\SetKwInOut{KwInOut}{In/out}
\KwInOut{$S$, the eigen space state}
\KwIn{$M$, the matrix in double precision}
\KwIn{$M_1$, the matrix in single precision}
\KwIn{$\chi$, the right hand side of the linear equation}
\KwIn{$n$, the maximum number of iterations}
\KwIn{$\epsilon$, required total precision}
\KwIn{$\epsilon_1$, required inner precision}
\KwOut{$\psi$, approximate solution}
\KwOut{$r$, final residue}
\KwOut{$k$, number of iterations used}
\KwOut{$z$, was the last $\pi$ a zero vector of $M$?}
\SetKw{Break}{break}
\SetKw{And}{and}
\SetKw{False}{false}
\SetKw{True}{true}
\SetKwFunction{innerCG}{innerCG}
\DontPrintSemicolon
\Begin{
  $\psi\leftarrow 0$\;
  $r\leftarrow \langle\chi,\chi\rangle$\;
  $\epsilon\leftarrow r\times\epsilon$\;
  $k\leftarrow 0$\;
  \While{$k < n$}{
    $\eta\leftarrow\chi - M^\dagger \times M \times \psi$\;
    $r\leftarrow\langle\eta,\eta\rangle$\;
    \lIf{$r<\epsilon$}{\Return{$\psi$, $r$, $k$, \False}}\;
    $\eta\leftarrow\eta/$sqrt$(r)$\;
    $\delta, q, m, z\leftarrow$\innerCG{$S$, $M_1$, $\eta$, $n-k$, $\max(\epsilon/r, \epsilon_1)$}\;
    \lIf{$z$ \And $m = 0$}{\Return{$\psi$, $r\times q$, $k$, \True}}\;
    $\psi\leftarrow \psi + $sqrt$(r)\times\delta$\;
    $k\leftarrow k + m$\;
  }
  $\eta\leftarrow\chi - M^\dagger \times M \times \psi$\;
  $r\leftarrow\langle\eta,\eta\rangle$\;
  \Return{$\psi$, $r$, $k$, \False}\;
}
\caption{eigenCG($S$, $M$, $M_1$, $\chi$, $n$, $\epsilon$, $\epsilon_1$)}
\end{function}

The \texttt{innerCG} is very similar to \texttt{plainCG}. There are two differences: (a) the initial values of $\rho$ and $\psi$ are obtained from $S$, and (b) $S$ is updated after every application of $M^\dagger M$.

\begin{function}
\SetKwInOut{KwInOut}{In/out}
\KwInOut{$S$, the eigen space state}
\KwIn{$M$, the matrix}
\KwIn{$\chi$, the right hand side of the linear equation}
\KwIn{$n$, the maximum number of iterations}
\KwIn{$\epsilon$, required precision}
\KwOut{$\psi$, approximate solution}
\KwOut{$r$, final residue}
\KwOut{$k$, number of iterations used}
\KwOut{$z$, was the last $\pi$ a zero vector of $M$?}
\SetKw{Break}{break}
\SetKw{And}{and}
\SetKw{False}{false}
\SetKw{True}{true}
\SetKwFunction{eigSpStart}{eigSpStart}
\SetKwFunction{eigSpUpdate}{eigSpUpdate}
\DontPrintSemicolon
\Begin{
  $\psi\leftarrow$\eigSpStart($S$, $\chi$)\;
  $\rho\leftarrow\chi-M^\dagger\times M \times \psi$\;
  $\pi\leftarrow \rho$\;
  $r\leftarrow\langle \rho,\rho\rangle$\;
  \lIf{$r = 0$}{
     \Return{$\psi$, 0, 0, \False}
  }\;
  $k\leftarrow 0$\;
  \While{$r > \epsilon$ \And $k < n$}{
    $\omega\leftarrow M\pi$\;
    $\zeta\leftarrow M^{\dagger}\omega$\;
    $c\leftarrow \langle \omega,\omega\rangle$\;
    \lIf{$c = 0$}{
      \Return{$\psi$, $r$, $k$, \True}
    }\;
    \eigSpUpdate{$S$, $\pi$, $\zeta$}\;
    $a\leftarrow r/c$\;
    $\rho\leftarrow \rho - a \zeta$\;
    $ g \leftarrow \langle \rho,\rho\rangle$\;
    $\psi\leftarrow \psi + a \pi$\;
    $k\leftarrow k + 1$\;
    \lIf{$ g<\epsilon$}{\Return {$\psi$, $g$, $k$, \False }} \;
    $b\leftarrow g/r$\;
    $r\leftarrow g$\;
    $\pi\leftarrow \rho + b \pi$\;
  }
  \Return {$\psi$, $r$, $k$, \False}
}
\caption{innerCG($S$, $M$, $\chi$, $n$, $\epsilon$)}
\end{function}

\section{Eigen space state}
Here are routines operating on the eigen space state $S$.
In the production implementation $S$ is likely to be a part of the solver state
\subsection{Constructor, \texttt{eigSpInit()}}
Initialize state variables and allocate space. The tolerance $\epsilon$ controls how small
a vector has to be to be considered a round-off error.
\begin{function}
\KwIn{$n$, the number of eigenvectors to keep}
\KwIn{$m$, the maximum dimension of the eigen space, $m > n$}
\KwIn{$\epsilon$, required precision}
\KwOut{$S$, the eigen space state}
\SetKw{False}{false}
\SetKw{True}{true}
\DontPrintSemicolon
\Begin{
 $S\leftarrow$ alloc structure\;
 $S.k\leftarrow 0$\;
 $S.n\leftarrow n$\;
 $S.m\leftarrow m$\;
 $S.\epsilon\leftarrow \epsilon$\;
 $S.blocked \leftarrow$\False\;
 $S.ready \leftarrow$\False\;
 $S.w\leftarrow$ alloc $n$ vectors\tcp*{space for eigenvectors $w$}\;
 $S.Aw\leftarrow$ alloc $n$ vectors\tcp*{space for $Aw$}\;
 $S.v\leftarrow$ alloc $m$ vectors\tcp*{space for basis vectors $v$}\;
 $S.Av\leftarrow$ alloc $m$ vectors\tcp*{space for $Av$}\;
 $S.H\leftarrow$ alloc $m\times m$ complex matrix\tcp*{space for $\langle v,Av\rangle$}\;
 $S.U\leftarrow$ alloc $m\times m$ complex matrix\tcp*{space for $H^{-1}$}\;
 $S.\lambda\leftarrow$ alloc $m$ reals\tcp*{space for eigenvalues}\;
 $S.\beta\leftarrow$ alloc $(m-1)$ reals\tcp*{space for off-diagonal}\;
  \Return {$S$}.
}
\caption{eigSpInit($n$, $m$, $\epsilon$)}
\end{function}

\subsection{Destructor, \texttt{eigSpFini()}}
Free all allocated space and free the structure
\begin{function}
\SetKwInOut{KwInOut}{In/out}
\KwInOut{$S$, the eigen space state}
\SetKw{False}{false}
\SetKw{True}{true}
\DontPrintSemicolon
\Begin{
 free $S.w$\;
 free $S.Aw$\;
 free $S.v$\;
 free $S.Av$\;
 free $S.H$\;
 free $S.U$\;
 free $S.\lambda$\;
 free $S.\beta$\;
 free $S$\;
}
\caption{eigSpFini($S$)}
\end{function}

\subsection{Update Activation, \texttt{eigSpEnable()}}
It is useful to be able to shut down and enable the updater.
\begin{function}
\SetKwInOut{KwInOut}{In/out}
\KwInOut{$S$, the eigen space state}
\SetKw{False}{false}
\SetKw{True}{true}
\DontPrintSemicolon
\Begin{
 $S.$blocked$\leftarrow$\False\;
}
\caption{eigSpEnable($S$)}
\end{function}

\subsection{Update Deactivation, \texttt{eigSpDisable()}}
It is useful to be able to shut down and enable the updater.
\begin{function}
\SetKwInOut{KwInOut}{In/out}
\KwInOut{$S$, the eigen space state}
\SetKw{False}{false}
\SetKw{True}{true}
\DontPrintSemicolon
\Begin{
 $S.$blocked$\leftarrow$\True\;
}
\caption{eigSpDisable($S$)}
\end{function}

\subsection{Starting the solver, \texttt{eigSpStart()}}
When $S$ is ready, the solver starter subtracts found eigenvectors and adjusts
the solution accordingly. For this routine to work, $S.Aw$ must be orthonormal, e.g.,
$\langle S.Aw[i], S.Aw[j]\rangle=\delta_{ij}$ and
$S.Aw[k]=M^\dagger M S.w[k]$ for each $k$.
\begin{function}
\KwIn{$S$, the eigen space state}
\KwIn{$\chi$, the right hand side}
\KwOut{$\psi$, the partial solution}
\SetKw{False}{false}
\SetKw{True}{true}
\DontPrintSemicolon
\Begin{
  $\psi\leftarrow 0$\;
  \If{$S.$ready}{
    \For(// remove eigen space parts from $\rho$ and $\psi$){$0\le k < S.n$}{
      $a\leftarrow\langle S.Aw[k], \chi\rangle$\;
      $\psi\leftarrow\psi-a\times S.w[k]$\;
    }
  }
  \Return{$\rho$, $\psi$}\;
}
\caption{eigSpStart($S$, $\chi$)}
\end{function}

\subsection{Updating the state, \texttt{eigSpUpdate()}}
Add a pair $\psi$, $\chi=M^\dagger M \psi$ to $S$. The external procedure \texttt{EigenV()} computes
the transformation to the eigenbasis with the eigenvalues in the increasing order.
\begin{function}
\SetKwInOut{KwInOut}{In/out}
\KwInOut{$S$, the eigen space state}
\KwIn{$\psi$, a new eigen space vector}
\KwIn{$\chi$, computed $M^\dagger M \psi$}
\SetKw{False}{false}
\SetKw{True}{true}
\SetKwFunction{eigSpBuildLow}{eigSpBuildLow}
\DontPrintSemicolon
\Begin{
  \lIf{S.blocked}{
    \Return{}\;
  }
  $k\leftarrow S.k$\;
  \tcp{add and normalize $\psi$ and $\chi$ to $S$}
  $S.v[k]\leftarrow\psi$\;
  $S.Av[k]\leftarrow\chi$\;
  $a\leftarrow\langle\psi, \psi\rangle$\;
  \For{$0 \le i < k$}{
    $b\leftarrow\langle S.v[i], S.v[k]\rangle$\;
    $c\leftarrow a-|b|^2$\;
    \tcp{check that there is something left still}
    \lIf{$c < a\times S.\epsilon$}{\Return{}\;}
    $S.v[k]\leftarrow  S.v[k] - b \times S.v[i]$\;
    $S.Av[k]\leftarrow S.Av[k] - b \times S.Av[i]$\;
    $a\leftarrow c$\;
  }
  $a\leftarrow\langle S.v[k], S.v[k]\rangle$\;
  $d\leftarrow 1/$sqrt$(a)$\;
  $S.v[k]\leftarrow d \times S.v[k] $\;
  $S.Av[k]\leftarrow d \times S.Av[k]$\;
  $S.k\leftarrow S.k + 1$\;
  \tcp{space is full. Find eigenvalues}
  \If{$S.k = S.m$}{
    \eigSpBuildLow{$S$}\;
    $S.k\leftarrow S.n$\;
    $S.ready\leftarrow\True$\;
  }
}
\caption{eigSpUpdate($S$, $\psi$, $\chi$)}
\end{function}

\begin{function}
\SetKwInOut{KwInOut}{In/out}
\KwInOut{$S$, the eigen space state}
\SetKw{False}{false}
\SetKw{True}{true}
\SetKwFunction{EigenT}{EigenT}
\SetKwFunction{eigSpOrtho}{eigSpOrtho}
\DontPrintSemicolon
\Begin{
    \tcp{compute $S.H$}
    \For{$0\le i < S.m$}{
       $S.H[i,i]\leftarrow $Re$\langle S.v[i],S.Av[i]\rangle$\;
       \For{$0\le j < i$}{
          $a\leftarrow\langle S.v[j],S.Av[i]\rangle$\;
          $S.H[j,i]\leftarrow a$\;
          $S.H[i,j]\leftarrow $conj$(a)$\;
       }
    }
    $S.U\leftarrow$\EigenT{$S.H$, $S$}\;
    \tcp{Compute $S.Aw$ and $S.w$}
    \For{$0\le i < S.n$}{
       $S.w[i]\leftarrow 0$\;
       $S.Aw[i]\leftarrow 0$\;
       \For{$0\le j < S.m$}{
         $S.w[i]\leftarrow S.w[i] + S.U[i.j] \times S.v[j]$\;
         $S.Aw[i]\leftarrow S.Aw[i] + S.U[i,j] \times S.Av[j]$\;
       }
    }
    \tcp{Copy $S.w$ and $S.Aw$ back to $S.v$ and $S.Av$}
    \For{$0\le i < S.n$}{
       $S.v[i]\leftarrow S.w[i]$\;
       $S.Av[i]\leftarrow S.Aw[i]$\;
    }
    \eigSpOrtho{$S$}\;
}
\caption{eigSpBuildLow($S$)}
\end{function}

\subsection{Recomputing the operator, \texttt{eigSpRecompute()}}
If too many linear algebra operations were performed, one might want to
recompute the operator.
\begin{function}
\SetKwInOut{KwInOut}{In/out}
\KwInOut{$S$, the eigen space state}
\KwIn{$M_1$, the right hand side}
\SetKw{False}{false}
\SetKw{True}{true}
\SetKwFunction{eigSpOrtho}{eigSpOrtho}
\DontPrintSemicolon
\Begin{
  \For(// compute $S.Av$){$0\le i < S.k$}{
    $S.Av[i] = M^\dagger \times M \times S.v[i]$\;
  }
  \If{$S.$ready}{
    \For(// compute $S.Aw$){$0\le i < S.n$}{
       $S.Aw[i] = M^\dagger \times M \times S.w[i]$\;
    }
    \eigSpOrtho{$S$}\;
    }
}
\caption{eigSpRecompute($S$, $M_1$)}
\end{function}

\begin{function}
\SetKwInOut{KwInOut}{In/out}
\KwInOut{$S$, the eigen space state}
\SetKw{False}{false}
\SetKw{True}{true}
\DontPrintSemicolon
\Begin{
    \For(// reorthogonalize $S.Aw$){$0\le i < S.n$}{
       \For{$0\le j < i$}{
          $a\leftarrow \langle S.Aw[j], S.Aw[i]\rangle$\;
          $S.Aw[i]\leftarrow S.Aw[i] - a \times S.Aw[j]$\;
          $S.w[i]\leftarrow S.w[i] - a \times S.w[j]$\;
       }
       $a \leftarrow \langle S.Aw[i], S.Aw[i]\rangle$\;
       $b\leftarrow 1/$sqrt $a$\;
       $S.Aw[i]\leftarrow b \times S.Aw[i]$\;
       $S.w[i]\leftarrow b \times S.w[i]$\;
   }
}
\caption{eigSpOrtho($S$)}
\end{function}

\end{document}
